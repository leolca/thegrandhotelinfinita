The paradox that Henrietta and Bernard tried to solve is an idea from the famous mathematician David Hilbert. He proposed a thought experiment: how can an infinite hotel, with every room occupied, accommodate a new guest? His solution was simple: move the guest in room $n$ to room $n+1$, which frees up room 1 for the new guest. To accommodate an infinite number of new guests, he suggested moving the guest in room $n$ to room $2n$, which frees up all the odd-numbered rooms for the new guests.

Our story extends this idea to two dimensions, creating a hotel with an infinite number of floors and infinite rooms on each floor. This makes the problem seem even more impossible! How could a single elevator visit every room, one by one?

Anne's solution, using the Cantor's Pairing Function, shows us something amazing about infinity. Intuitively, it seems like a hotel with an infinite number of floors and an infinite number of rooms per floor ($\mathbb{N}\times\mathbb{N}$) would be much, much bigger than just a single infinite list of numbers ($\mathbb{N}$) that represents the freight elevator's position on its infinite track to visit the rooms. But the Cantor function shows a bijection--a one-to-one correspondence--between these two sets, proving that they are the same size.

The mathematician Georg Cantor proved that these two infinities are actually the same size. He called this ``size'' cardinality, and the cardinality of a countably infinite set (like the natural numbers) is called Aleph-zero ($\aleph_0$). The function Anne used performs the magic: it takes a pair of numbers (like the room and floor number, $(i,j$)) and turns it into a single unique number (the elevator's position, n). 
$$n=(i+j)(i+j+1)/2+j$$
What's even more incredible is that it can be reversed to find the original pair.
The mapping used by Anne is shown in the picture below:

\begin{center}
%\documentclass[tikz,border=2mm]{standalone}
%\usepackage{amsmath}
%\usepackage{tikz}
%\usetikzlibrary{arrows.meta, decorations.pathmorphing}
%\begin{document}
\begin{tikzpicture}[
    >=Latex, 
    node style/.style={circle, fill=red, inner sep=2pt, font=\bfseries\sffamily, text=white},
    label style/.style={font=\sffamily},
    grid style/.style={line width=0.5pt, gray!40},
    path style/.style={line width=1.5pt, blue, ->},
    dashed path style/.style={line width=1.5pt, blue, dashed, ->}
]

% Draw the grid
\draw[grid style] (0,0) grid (4,4);

% Draw the axes and labels
\draw[->] (0,0) -- (4.5,0) node[right, label style] {i (room number)};
\draw[->] (0,0) -- (0,4.5) node[above, label style] {j (floor number)};

% Draw the nodes for the grid points
\foreach \x in {0,1,2,3,4} {
    \foreach \y in {0,1,2,3,4} {
        \node[node style, label={[label style]above right:}] at (\x, \y) {};
    }
}

% Add labels for the axes values
\foreach \i in {0,1,2,3,4} {
    \node[below=15pt, label style] at (\i,0) {\i};
}
\foreach \j in {0,1,2,3,4} {
    \node[left=10pt, label style] at (0,\j) {\j};
}

% Define the path based on the Cantor pairing function
% Path for n=0 to n=10
%\draw[path style] (0,0) -- node[above, pos=0.5, font=\sffamily] {0} (0,0.1);
\draw[path style] (0,0) -- (1,0) node[below, pos=0, font=\sffamily] {0};
\draw[path style] (1,0) -- (0,1) node[below, pos=0, font=\sffamily] {1};
\draw[path style] (0,1) -- (2,0) node[left, pos=0, font=\sffamily] {2};
\draw[path style] (2,0) -- (1,1) node[below, pos=0, font=\sffamily] {3};
\draw[path style] (1,1) -- (0,2) node[left, pos=0, font=\sffamily] {4};
\draw[path style] (0,2) -- (3,0) node[left, pos=0, font=\sffamily] {5};
\draw[path style] (3,0) -- (2,1) node[below, pos=0, font=\sffamily] {6};
\draw[path style] (2,1) -- (1,2) node[right, pos=0.05, font=\sffamily] {7};
\draw[path style] (1,2) -- (0,3) node[left, pos=0, font=\sffamily] {8};
\draw[path style] (0,3) -- (4,0) node[left, pos=0, font=\sffamily] {9};
\draw[path style] (4,0) -- (3,1) node[below, pos=0, font=\sffamily] {10};
\draw[dashed path style] (3,1) -- (2,2) node[right, pos=0.05, font=\sffamily] {11};
\draw[dashed path style] (2,2) -- (1,3);% node[above, pos=0.5, font=\sffamily] {13};
\draw[dashed path style] (1,3) -- (0,4);% node[above, pos=0.5, font=\sffamily] {14};
%\draw[path style] (0,4) -- (4,1) node[above, pos=0.5, font=\sffamily] {15};
%\draw[path style] (4,1) -- (3,2) node[above, pos=0.5, font=\sffamily] {16};
%\draw[path style] (3,2) -- (2,3) node[above, pos=0.5, font=\sffamily] {17};
%\draw[path style] (2,3) -- (1,4) node[above, pos=0.5, font=\sffamily] {18};
%\draw[path style] (1,4) -- (4,2) node[above, pos=0.5, font=\sffamily] {19};
%\draw[path style] (4,2) -- (3,3) node[above, pos=0.5, font=\sffamily] {20};
%\draw[path style] (3,3) -- (2,4) node[above, pos=0.5, font=\sffamily] {21};
%\draw[path style] (2,4) -- (4,3) node[above, pos=0.5, font=\sffamily] {22};
%\draw[path style] (4,3) -- (3,4) node[above, pos=0.5, font=\sffamily] {23};
%\draw[path style] (3,4) -- (4,4) node[above, pos=0.5, font=\sffamily] {24};
\end{tikzpicture}
%\end{document}


\end{center}

%This one-to-one correspondence means that the set of all pairs of rooms and floors is, in fact, the same ``size'' as the set of all natural numbers. It's a concept that's a little hard to wrap your head around, but it's one of the great secrets of infinity!

This one-to-one correspondence means that the set of all pairs of rooms and floors is, in fact, the same ``size'' as the set of all natural numbers. This challenges one of our most basic intuitions. As the mathematician David Hilbert pointed out, when dealing with finite sets, a part is always smaller than the whole. But when we are dealing with infinite sets, this principle no longer applies. As he said: ``Hier gilt nun schon der Satz: ,,Der Teil ist kleiner als das Ganze'' nicht mehr.'' (Hilbert, 1925 in lecture ``\"{U}ber das Unendliche''). It's a concept that's a little hard to wrap your head around, but it's one of the great secrets of infinity!
