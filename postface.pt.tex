O paradoxo que Henrietta e Bernard tentaram resolver é uma ideia do famoso matemático David Hilbert. Ele propôs um experimento mental: como um hotel infinito, com todos os quartos ocupados, pode acomodar um novo hóspede? A solução dele era simples: mude o hóspede do quarto $n$ para o quarto $n+1$, o que libera o quarto 1 para o novo hóspede. Para acomodar um número infinito de novos hóspedes, ele sugeriu mover o hóspede do quarto $n$ para o quarto $2n$, o que libera todos os quartos de número ímpar para os novos hóspedes.

Nossa história estende essa ideia para duas dimensões, criando um hotel com um número infinito de andares e infinitos quartos em cada andar. Isso faz com que o problema pareça ainda mais impossível! Como um único elevador poderia visitar cada quarto, um por um?

A solução de Anne, usando a função de pareamento de Cantor, nos mostra algo incrível sobre o infinito. Intuitivamente, parece que um hotel com um número infinito de andares e um número infinito de quartos por andar ($\mathbb{N}\times\mathbb{N}$) seria muito, muito maior do que uma única lista infinita de números ($\mathbb{N}$) que representa a posição do elevador de carga em seu trilho infinito para visitar os quartos. Mas a função de Cantor mostra uma bijeção -- uma correspondência um-para-um -- entre esses dois conjuntos, provando que eles têm o mesmo tamanho.

O matemático Georg Cantor provou que esses dois infinitos são, na verdade, do mesmo tamanho. Ele chamou essa ``tamanho'' de cardinalidade, e a do infinito contável (como os números naturais) é chamada de Aleph-zero ($\aleph_0$). A função que Anne usou faz a mágica: ela pega um par de números (como o número do quarto e do andar, $(i,j)$) e o transforma em um único número (a posição do elevador, $n$). 
$$n=(i+j)(i+j+1)/2+j$$
O que é ainda mais incrível é que ela pode ser revertida para encontrar o par original. O mapeamento usado por Anne é mostrado na figura abaixo:

\begin{center}
%\documentclass[tikz,border=2mm]{standalone}
%\usepackage{amsmath}
%\usepackage{tikz}
%\usetikzlibrary{arrows.meta, decorations.pathmorphing}
%\begin{document}
\begin{tikzpicture}[
    >=Latex, 
    node style/.style={circle, fill=red, inner sep=2pt, font=\bfseries\sffamily, text=white},
    label style/.style={font=\sffamily},
    grid style/.style={line width=0.5pt, gray!40},
    path style/.style={line width=1.5pt, blue, ->},
    dashed path style/.style={line width=1.5pt, blue, dashed, ->}
]

% Draw the grid
\draw[grid style] (0,0) grid (4,4);

% Draw the axes and labels
\draw[->] (0,0) -- (4.5,0) node[right, label style] {i (número do quarto)};
\draw[->] (0,0) -- (0,4.5) node[above, label style] {j (número do andar)};

% Draw the nodes for the grid points
\foreach \x in {0,1,2,3,4} {
    \foreach \y in {0,1,2,3,4} {
        \node[node style, label={[label style]above right:}] at (\x, \y) {};
    }
}

% Add labels for the axes values
\foreach \i in {0,1,2,3,4} {
    \node[below=15pt, label style] at (\i,0) {\i};
}
\foreach \j in {0,1,2,3,4} {
    \node[left=10pt, label style] at (0,\j) {\j};
}

% Define the path based on the Cantor pairing function
% Path for n=0 to n=10
%\draw[path style] (0,0) -- node[above, pos=0.5, font=\sffamily] {0} (0,0.1);
\draw[path style] (0,0) -- (1,0) node[below, pos=0, font=\sffamily] {0};
\draw[path style] (1,0) -- (0,1) node[below, pos=0, font=\sffamily] {1};
\draw[path style] (0,1) -- (2,0) node[left, pos=0, font=\sffamily] {2};
\draw[path style] (2,0) -- (1,1) node[below, pos=0, font=\sffamily] {3};
\draw[path style] (1,1) -- (0,2) node[left, pos=0, font=\sffamily] {4};
\draw[path style] (0,2) -- (3,0) node[left, pos=0, font=\sffamily] {5};
\draw[path style] (3,0) -- (2,1) node[below, pos=0, font=\sffamily] {6};
\draw[path style] (2,1) -- (1,2) node[right, pos=0.05, font=\sffamily] {7};
\draw[path style] (1,2) -- (0,3) node[left, pos=0, font=\sffamily] {8};
\draw[path style] (0,3) -- (4,0) node[left, pos=0, font=\sffamily] {9};
\draw[path style] (4,0) -- (3,1) node[below, pos=0, font=\sffamily] {10};
\draw[dashed path style] (3,1) -- (2,2) node[right, pos=0.05, font=\sffamily] {11};
\draw[dashed path style] (2,2) -- (1,3);% node[above, pos=0.5, font=\sffamily] {13};
\draw[dashed path style] (1,3) -- (0,4);% node[above, pos=0.5, font=\sffamily] {14};
%\draw[path style] (0,4) -- (4,1) node[above, pos=0.5, font=\sffamily] {15};
%\draw[path style] (4,1) -- (3,2) node[above, pos=0.5, font=\sffamily] {16};
%\draw[path style] (3,2) -- (2,3) node[above, pos=0.5, font=\sffamily] {17};
%\draw[path style] (2,3) -- (1,4) node[above, pos=0.5, font=\sffamily] {18};
%\draw[path style] (1,4) -- (4,2) node[above, pos=0.5, font=\sffamily] {19};
%\draw[path style] (4,2) -- (3,3) node[above, pos=0.5, font=\sffamily] {20};
%\draw[path style] (3,3) -- (2,4) node[above, pos=0.5, font=\sffamily] {21};
%\draw[path style] (2,4) -- (4,3) node[above, pos=0.5, font=\sffamily] {22};
%\draw[path style] (4,3) -- (3,4) node[above, pos=0.5, font=\sffamily] {23};
%\draw[path style] (3,4) -- (4,4) node[above, pos=0.5, font=\sffamily] {24};
\end{tikzpicture}
%\end{document}


\end{center}

%Essa correspondência de um-para-um significa que o conjunto de todos os pares de quartos e andares é, de fato, do mesmo ``tamanho'' que o conjunto de todos os números naturais. É um conceito um pouco difícil de engolir, mas é um dos grandes segredos do infinito!

Essa correspondência um-para-um significa que o conjunto de todos os pares de quartos e andares é, na verdade, do mesmo ``tamanho'' que o conjunto de todos os números naturais. Isso desafia uma de nossas intuições mais básicas. Como o matemático David Hilbert apontou, ao lidar com conjuntos finitos, uma parte é sempre menor do que o todo. Mas, ao lidar com conjuntos infinitos, esse princípio não se aplica mais. Como ele disse: ``Hier gilt nun schon der Satz: ,,Der Teil ist kleiner als das Ganze'' nicht mehr.'' (Hilbert, 1925 em sua palestra ``\"{U}ber das Unendliche''). É um conceito um pouco difícil de assimilar, mas é um dos grandes segredos do infinito!
